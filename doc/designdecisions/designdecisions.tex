% Designdecisions for ui2go.
%
% For PDF output: pdflatex designdecisions.tex
% For HTML output: plastex designdecisions.tex
%
% Graphics:
% Use PNG format to avoid problems with HTML conversion.
% Recommended size: < 13cm x 18cm (width x height).
% Recommended resolution: 300 dpi.
%
% Use fixed width instead of textwidth, so that plastex can
% recognize the graphics size. For example use
% \includegraphics[width=13cm]{img/haplogroups.png}
% instead of
% \includegraphics[width=\textwidth]{img/haplogroups.png}

\documentclass[12pt,a4paper]{article}
\usepackage[utf8]{inputenc}
\usepackage[english]{babel}
\usepackage[colorlinks=true, urlcolor=blue, linkcolor=blue]{hyperref}
\usepackage{graphicx}

\begin{document}
\begin{titlepage}

\title{ui2go White Paper}

\author{Dirk Struve\\
phylofriend at projectory.de\\
\href{https://github.com/yogischogi/ui2go/}{https://github.com/yogischogi/ui2go/}}
\date{\today}
\end{titlepage}
\maketitle

\tableofcontents


\section{Why ui2go?}

\subsection{Programming Situation}

\begin{enumerate}
\item
  Started learning Korean and wrote a small
  \href{http://www.projectory.de/koreanisch/index.html}{language course}.
\item
  Needed several programs to automate language course building.
\item
  No time for programming, wanted to learn Korean (still want to, spent
  too much time on programming ;-).
\item
  As I just spent a few hours every month on programming, the following
  issues became most important:

  \begin{itemize}
  \item
    Readability: After a month I forgot most things I had done before.
  \item
    Productivity: I wanted to spend time on programming, not on
    programming languages and associated tools.
  \item
    Incremental programming: I had no time to do big changes at once.
  \item
    Stability of programming environment: Widely adopted programming
    tools often change and require lots of maintenance. This usually
    goes unnoticed when using them daily.
  \end{itemize}
\item
  Go turned out to be a good choice for the task.
\item
  But I needed a simple UI for some tasks sometimes.
\item
  Existing UI technologies left me frustrated.
\end{enumerate}

\subsection{What I Needed}

\begin{enumerate}
\item
  Fast creation of simple UIs.
\item
  No fully fledged UI toolkit.
\end{enumerate}


\section{Events}

\subsection{Event Propagation}

\begin{enumerate}
\item
  Event system should be useful for arbitrary programs.
\item
  Event package contains universal event sender and receiver classes.
\item
  Every class can be made an event sender or receiver just by mixing in
  event sender or receiver.
\item
  Event senders and receivers support function calls and channels for
  event propagation.
\item
  Programmers can use push or pull style.
\item
  In GUI programs: circular control flow metaphor

  \begin{itemize}
  \item
    Events flow from input device to GUI, from there to the program
    logic and back again.
  \item
    automatic event propagation from window down to the controls and up
    again
  \item
    automatic event propagation done by function calls (no concurrency
    issues, much faster than channels)
  \end{itemize}
\end{enumerate}

\subsection{Event Structure}

\begin{enumerate}
\item
  Traditional event systems are created for performance and hard to use.
\item
  Try to create easy-to-use and semantically meaningful events.
\end{enumerate}


\section{Layout Management}

\begin{enumerate}
\item
  Inspired by MiG-Layout, but simpler and easier.
\item
  Layout is done like printing lines (a bit like like Printf).
\item
  Layout manager tries to take the burden from the programmer.
\item
  Mock-up mode for testing a layout without creating widgets.
\item
  No pixel accurate layout (layout manager tries to automate as much as
  possible).
\end{enumerate}


\section{Drawing Model}

\begin{enumerate}
\item
  All drawing operations are based upon
  \href{https://github.com/ungerik/go-cairo}{go-cairo}.
\item
  Cairo is stable, widely used and well documented.
\item
  But cairo is not "goish", feels a bit strange sometimes.
\item
  Drawing of widgets (not layout) like CSS box model.
\end{enumerate}


\section{Cross Platform}

\subsection{Pros}

\begin{enumerate}
\item
  nice to have the same API on different platforms
\item
  results in larger user group
\item
  will increase project popularity
\end{enumerate}

\subsection{Cons}

\begin{enumerate}
\item
  A cross platform library does not make a program cross platform.

  \begin{itemize}
  \item
    lots of subtle platform specific details (apart from the library),
    that often require an extreme amount of work for simple tasks
  \item
    truly platform independence results in middle-ware OS
  \end{itemize}
\item
  Library will become more complex (error prone and slow).
\item
  Dependencies on extra libraries introduce lots of problems (bugs,
  version changes).
\item
  Cross platform design puts restrictions on the project that may
  turn out as big problems later.
\item
  Burden to the programmer.

  \begin{itemize}
  \item
    longer training period because of extra complexity
  \item
    Programmers need to know the details of the cross platform API and
    the underlying platform (marketing people always make different
    claims, but...).
  \end{itemize}
\item
  Platform specific programs tend to be smaller, faster and more usable.
\end{enumerate}

\subsection{Solution}

\begin{enumerate}
\item
  The need for an easy-to-use event system turned the decision in favour
  of a cross platform solution.
\item
  Window abstraction layer directly on top of the system libraries.
  \begin{itemize}
  \item
    small code base
  \item
    fast
  \end{itemize}
\end{enumerate}

\end{document}



