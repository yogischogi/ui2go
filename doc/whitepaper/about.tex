\section{About ui2go}

\subsection{What is ui2go?}

ui2go is a GUI library for the Go programming language. It should enable
the programmer to create graphical user interfaces in an easy way. At
the moment ui2go is in a very early stage of development and practically
nothing more than a showcase.

The project is hosted at \url{https://github.com/yogischogi/ui2go/}.

\subsection{Why was ui2go created?}

I started this project out of frustration. Although there is an ongoing
fuzz about new user interfaces, programming technologies themselves
doesn't seem to make much progress. Even worse, I can't help the
feeling, that creating graphical user interfaces has become more
complicated and time consuming than ever before.

So I asked myself, if there was a better way and started to explore
different alternatives. ui2go is a first prototype of my thoughts.

\subsection{Why not use some Web technology?}

Well, the answer to that question is a bit complicated. I went to an
expert to ask him for advice and this is what happened :-)

\begin{longtable}{lp{11cm}}
Me: & I want to write a program with a graphical user interface.\\
Expert: & Sure, nothing easier than that. Just use some Web technology.\\
Me: & Isn't it a bit too complicated?\\
Expert: & No, not at all! You just install a web server, write some HTML
code and point your browser to it.\\
Me: & Are you telling me, that all I need is to install and configure a
web server, write some HTML and use my web browser to view the result?\\
Expert: & Yeah, easy, isn't it?\\
Me: & But isn't it difficult to organize all these HTML pages?\\
Expert: & No, not at all. There is this thing called Big Typo. It makes
things really easy.\\
Me: & Typo? Isn't that another word for mistake?\\
Expert: & Yeah, but that's the way people write on the web these days.\\
Me: & To be honest. I tried HTML before and found it quite hard to get
the layout right.\\
Expert: & No need to worry about that. Use the jWeary library! It makes
things really simple and reduces the amount of HTML code a lot.\\
Me: & So I only need to learn HTML and this j-scripting stuff?\\
Expert: & Yeah, that's really easy.\\
Me: & Actually I have tried this scripting stuff before and it just
doesn't seem to work out right.\\
Expert: & Oh, you probably used the wrong browser. What are you using?\\
Me: & I like Excalibur a lot.\\
Expert: & See? I guessed right. You don't want to use Excalibur. These
guys have microscopic brains and are a bit soft in the head. You surely
don't want to be like them.\\
Me: & Well, what about using Burning Tails instead?\\
Expert: & Yeah, guys with Burning Tails are fast, but it's painful
sometimes. You better use Nice n' Shine to stay out of trouble.\\
Me: & I have got another problem. Whenever I use some kind of scripting
language my programs tend to get a bit tangled when they get bigger.\\
Expert: & This scripting stuff might not be for everyone. But don't
worry. There is an extremely simple solution for that. Use Smart! I's a
brand new programming language and even compiles to scripts. You surely
hit your target with this one! Won't get any easier, I can tell you
that.\\
Me: & So I only need to install and configure a web server, use this Big
Mistake and the right web browser, install the Smart stuff and this
jWeary, learn some HTML, scripting and Smart. Is this really all there
is to it?\\
Expert: & Yeah, I told you it's easy, right?\\
Me: & But what if I need access to some operating systems library?\\
Expert: & I tell you a secret. There is this awesome Holy Grails
framework. It makes things even simpler.\\
Me: & I'm not really sure about that.\\
Expert: & I show you a program, I've written in just a couple of hours.
Look!\\
Me: & Well,\dots that's \emph{Hello World!}\\
Expert: & Yeah, but it's kind of cool, isn't it?
\end{longtable}

Apart from the fun I believe that Web solutions are not well suited for
a lot of serious programs. Here are some arguments:

\begin{enumerate}
\item
  Web programs require a complex multilayer architecture (operating
  system, web browser, scripting libraries) to work.

  The resulting system is overly complex, hard to maintain, error prone,
  slow and creates lots of potential security risks.
\item
  Web solutions are slow and energy consuming. This is certainly not be
  the best contribution to green computing.
\item
  Web programs are a security risk. There are many software components
  involved and the browser is the one program, that is exposed most to
  the dangerous world.

  People, who take Web security seriously, often end up with some kind
  of multi browser multi configuration environment, that is costly and
  hard to maintain.
\end{enumerate}

I do not doubt the tremendous benefits of the Web, but one should always
keep in mind, that it might not be the best platform for a lot of
applications it is frequently used for.

