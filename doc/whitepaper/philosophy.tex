\section{Software Design Philosophies}

When creating a software system there are always a lot of design
decisions to make. Usually a specific solution is chosen by using a set
of metrics based upon some kind of philosophy. Because different people
value different things there are different design philosophies.

People with different philosophies usually don't get along very well and
there are a lot of misunderstandings, just because they use different
metrics.

So it is important to take a look at some design approaches to software
development. There are many, but these are some common ones:

\subsection{Possibility Approach}

In many cases people don't consider advantages or disadvantages of a
design. In fact, they don't design at all. They just code, trying to
achieve some goal somehow on the path of least resistance.

Such people often point out that their design is straightforward, simple
and easy. Their biggest argument is that something is possible
(\emph{Look! It is possible to do this and that. So it must be good!}).
If something is possible somehow, many people stop thinking and don't
ask themselves if there is a better way to do it.

The possibility approach often works quite well for smaller programs,
but when programs get bigger design problems start to emerge.

\subsubsection{Metrics}

\begin{enumerate}
\item
  Is it possible to create a solution in this way?
\end{enumerate}

\subsubsection{Frequently attributed as}

\begin{enumerate}
\item
  hands on
\item
  straight forward
\item
  simple, easy
\item
  pragmatic
\item
  quick and dirty
\end{enumerate}

\subsection{Complexity or Big Boots Approach}

Most people tackle complexity by adding even more complexity. They add
software layers, libraries and tools to make things easier, but in the
end it turns out, that the overall complexity got even bigger.

So why is this called the \emph{Big Boots Approach}? Imagine you are
hiking and get lost in a swamp. Now you could either admit, that you
took the wrong way, turn around and walk some steps back or you could
put on your big boots and continue walking, highly praising your boots.
Due to some mysterious reasons in software industry wearing big boots is
actually considered as being smart. So it might be a good idea wearing
them :-) (There is also a nice SpongeBob episode called \emph{Squeaky
Boots}.)

\subsubsection{Metrics}

\begin{enumerate}
\item
  Lots of commands
\item
  Shortcuts for every task anyone could imagine.
\end{enumerate}

\subsubsection{Frequently attributed as}

\begin{enumerate}
\item
  powerful
\item
  professional
\item
  works like magic (no one knows why magic works and if it works at all)
\end{enumerate}

\subsection{Usability Approach}

I personally believe that legibility is one of the most important
aspects in writing code because usually most of the time is spend
maintaining the code and not writing it. In bigger projects writing a
piece of code and maintaining it is often done by different people. So
legibility comes to be even more important.

One way to write legible software is using the usability approach.
Unfortunately when it comes to usability everybody has their own
opinion. Many are surprised, that usability is actually quite well
defined and based upon psychological research. Usable code satisfies the
following criteria:

\subsubsection{Metrics}

\begin{enumerate}
\item
  simple
\item
  efficient
\item
  predictable
\item
  fault tolerant
\item
  transparent
\end{enumerate}

\subsubsection{Frequently attributed as}

\begin{enumerate}
\item
  easy
\item
  user/programmer friendly
\item
  less complex
\item
  oversimplified
\end{enumerate}

ui2go tries to implement usable code, because I believe that usable code
is more important than simple, elegant or powerful code. However, often
it just turns out to be simple, elegant and powerful.

